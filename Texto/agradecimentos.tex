\begin{agradecimentos}
	
	Aos meus pais e meu irmão, aos meus tios, João e Simone, e à Aline, minha ``prima-irmã'' e afilhada. Enfim, a toda minha primeira e grande família: ensinamentos e valores que transcendem em muito a importância de títulos acadêmicos.
	
	À minha segunda família: mãe Zélia, irmãos Rodrigo, Juliana, Bruno, Iara, Ariel, Karen e tantos outros, que foram se agregando, compartilhando e tornando a jornada mais fácil. Ao engrandecimento que me trouxeram.
	
	À Petúnia, pelo amor, incentivo e uma carinhosa paciência, difíceis de traduzir em palavras. Também aos seus pais, Wilson e Derci, pelo acolhimento e exemplo.
	
	Aos amigos da UFBA, melhor legado que levo desse curso: Simon, Jivago, Adinaílson, Ana Rita, Túlio, Fred, Roberto e Tássio. Poderia discorrer muito sobre a contribuição que cada um deu nestes longos anos de graduação, mas a beleza está na cumplicidade da amizade.
	
	Aos professores - concordando ou não com suas metodologias, todo aprendizado adquirido foi válido.
	
	Aos orientadores Edson e Ana Isabela, meu mais profundo apreço pelo auxílio, desde 2011, na iniciação científica, até agora, no presente trabalho.
	
\end{agradecimentos}