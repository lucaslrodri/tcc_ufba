%%%
%%% CARREGANDO PACOTES
%%%

\documentclass[
	openright, % Capítulos começam em página direita
	twoside,   % Impressão frente e verso
	a4paper,   % Tamanho do papel
	english,   % Idioma adiciona para hifenizaçãoo
	brazil     % O ultimo é o idioma principal
]{abntex2}

\usepackage{lmodern}                               % Usa família de fontes Latin Modern
\renewcommand*\familydefault{\sfdefault}           % Usa fonte Latin Modern Sans (Sem serifa)
\usepackage[T1]{fontenc}                           % Codifição Type1 para a fonte (Fontes de 8 bits para incluir as fontes do alfabeto brasileiro)
\usepackage[UTF8]{inputenc}                        % Codificação unicode "UTF8"
\usepackage{microtype}                             % Melhorias na justificação (Evita bad/underfullbox)
\usepackage{indentfirst}                           % Indenta o primeiro parágrafo da seção
%\usepackage{lastpage}                              % Conta o número total de páginas (P/ Ficha catalográfica)

\usepackage{multirow}                              % Suporte a células de tabelas com multilinhas
\usepackage{amssymb}                               % Símbolos matemáticos
\usepackage{enumitem}                              % Melhoria no suporte aos ambientes de numeração

\usepackage{color}                                 % Suporte a cores
\definecolor{blue}{RGB}{41,5,195}                  % Alterando o aspécto da cor Azul
\usepackage{graphicx}                              % Suporte a gráficos e figuras

\usepackage[alf]{abntex2cite}                      % Citações no padrão ABNT



%%%
%%% NOVOS COMANDOS
%%%

\renewcommand{\imprimircapa}{
	\begin{capa}%
		\center
		\begin{minipage}{1.0\textwidth}
			\begin{minipage}[c]{0.14\textwidth}
				\centering
				\includegraphics[width=0.8\textwidth]{figuras/brasao_ufba.jpg}
			\end{minipage}
			\hfill
			\begin{minipage}[c]{0.7\textwidth}
				\centering
				\ABNTEXchapterfont\large\textbf{\MakeUppercase\imprimirinstituicao}
			\end{minipage}
			\hfill
			\begin{minipage}[c]{0.14\textwidth}
				\includegraphics[width=0.9\textwidth]{figuras/brasao_poli.jpg}
			\end{minipage}
		\end{minipage}
		\\
		\vspace*{2cm}
		{\ABNTEXchapterfont\large\MakeUppercase\imprimirautor}
		\vfill
		{\ABNTEXchapterfont\bfseries\LARGE\MakeUppercase\imprimirtitulo}
		\vfill
		{\large\MakeUppercase\imprimirlocal} \\
		{\large\MakeUppercase\imprimirdata}
		\vspace*{1cm}
	\end{capa}
}

\renewcommand{\imprimirfolhaderosto}{
	\begin{folhaderosto}%
		\centering
		{\ABNTEXchapterfont\large\textbf{\MakeUppercase\imprimirinstituicao}}\\
		\vspace*{2cm}
		{\ABNTEXchapterfont\large\MakeUppercase\imprimirautor}\\
		\vfill
		{\ABNTEXchapterfont\bfseries\LARGE\MakeUppercase\imprimirtitulo}
		\vspace*{0.5cm}
		\begin{center}
			\hspace{.35\textwidth}
			\begin{minipage}{.5\textwidth}
				\large\imprimirpreambulo
				\vspace*{0.5cm}
			\end{minipage}%
		\end{center}
		{\large Orientador: Prof. Dr. \imprimirorientador\\
		\vspace*{\fill}}
		{\large\MakeUppercase\imprimirlocal} \\
		{\large\MakeUppercase\imprimirdata}
		\vspace*{1cm}
	\end{folhaderosto}
}

\newcommand{\imprimirfolhadeaprovacao}[2]{
	\begin{folhadeaprovacao}%
		\centering
		{\ABNTEXchapterfont\large\textbf{\MakeUppercase{Termo de Aprovação}}}\\
		\vspace*{0.5cm}
		{\ABNTEXchapterfont\large\MakeUppercase\imprimirautor}\\
		\vspace*{4cm}
		{\ABNTEXchapterfont\bfseries\LARGE\MakeUppercase\imprimirtitulo}
		\vspace*{0.5cm}
		\begin{center}
			\hspace{.35\textwidth}
			\begin{minipage}{.5\textwidth}
				\large
				Este Trabalho de Graduação foi julgado adequado para a obtenção do grau de Engenheiro Eletricista e aprovado em sua forma final pela Comissão Examinadora e pelo Colegiado do Curso de Graduação em Engenharia Elétrica da Universidade Federal da Bahia.
				\vspace*{0.5cm}
			\end{minipage}%
		\end{center}
		\vfill
		\assinatura{Prof. Dr. \imprimirorientador \\Orientador\\}
		\assinatura{#1 \\ Examinador\\}
		\assinatura{#2 \\ Examinador}
		\vspace*{2cm}
		{\large\MakeUppercase\imprimirlocal} \\
		{\large\MakeUppercase\imprimirdata}
		\vspace*{1cm}
	\end{folhadeaprovacao}
}

\newcommand{\imprimirdedicatoria}[1]{
	\begin{dedicatoria}
		\vspace*{\fill}
		\centering
		\noindent
		\textit{#1}
		\vspace*{\fill}
	\end{dedicatoria}
}

\newcommand{\imprimirepigrafe}[2]{
	\begin{epigrafe}
		\vspace*{\fill}
		\begin{flushright}
			\textit{``#1''\\
				(#2)}
		\end{flushright}
	\end{epigrafe}
}

%%%
%%% Dados do trabalho
%%%

\instituicao{Universidade Federal da Bahia\par
	Escola Politécnica\par
	Curso de Graduação em Engenharia Elétrica
}

\tipotrabalho{Trabalho Final de Graduação}

\preambulo{Trabalho apresentado ao Curso de Graduação em Engenharia Elétrica da Universidade Federal da Bahia como parte dos requisitos para a obtenção do título de Engenheiro Eletricista.}


% informações do PDF
\makeatletter
\hypersetup{
	%pagebackref=true,
	pdftitle={\@title}, 
	pdfauthor={\@author},
	pdfsubject={\imprimirpreambulo},
	pdfcreator={LaTeX with abnTeX2},
	pdfkeywords={abnt}{latex}{abntex}{abntex2}{trabalho acadêmico}, 
	colorlinks=true,       		% false: boxed links; true: colored links
	linkcolor=black,          	% color of internal links
	citecolor=black,        		% color of links to bibliography
	filecolor=black,      		% color of file links
	urlcolor=black,
	bookmarksdepth=4
}
\makeatother

% O tamanho do parágrafo é dado por:
\setlength{\parindent}{1.3cm}

% Controle do espaçamento entre um parágrafo e outro:
\setlength{\parskip}{0.2cm}  % tente também \onelineskip
